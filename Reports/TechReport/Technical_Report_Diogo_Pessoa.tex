\documentclass[12pt, a4paper]{article}
\usepackage{hyperref}
\usepackage{blindtext}
\usepackage{graphicx}
\setlength{\oddsidemargin}{0.5cm}
\setlength{\evensidemargin}{0.5cm}
\setlength{\topmargin}{-1.6cm}
\setlength{\leftmargin}{0.5cm}
\setlength{\rightmargin}{0.5cm}
\setlength{\textheight}{24.00cm}
\setlength{\textwidth}{15.00cm}
\parindent 0pt
\parskip 5pt
\pagestyle{plain}

%% Bibliography Management
%\usepackage[backend=biber, style=authoryear, natbib=true]{biblatex}
%\addbibresource{references.bib} % Your bibliography file

\title{Technical Report Assignment 1}
\author{Diogo Pessoa}
\date{07/02/2024}

\newcommand{\namelistlabel}[1]{\mbox{#1}\hfil}
\newenvironment{namelist}[1]{%1
    \begin{list}{}
    {
        \let\makelabel\namelistlabel
        \settowidth{\labelwidth}{#1}
        \setlength{\leftmargin}{1.1\labelwidth}
    }
    }{%1
    \end{list}}

\begin{document}
    \maketitle
    \begin{namelist}{xxxxxxxxxxxx}
        \item[\textbf{Title:}]
        The Divvy Bike ride-sharing Exploratory Data Analysis
        \item[\textbf{Author:}]
        Diogo Pessoa
        \item[\textbf{Degree:}]
        Postgraduate in Big Data Analytics and Artificial Intelligence
    \end{namelist}

    \section*{Problem Description}
    \label{sec:ProblemDescription}
    Bike ride-sharing apps confront the challenge of optimising bike availability across their stations, a dilemma that intensifies near tourist attractions and along commuter routes.
    Chiariotti et al.(2018)\cite{s18020512} underscore the significant imbalance within the system,
    particularly the rapid depletion of bicycles at residential stations in the morning and congestion at commercial area stations.

    This report systematically examines bike share trip records from the publicly sourced Divvy bike share system~\cite{DataSource}.
    By deriving and exploring additional features from historical trip data, this study aims to uncover trends in user behaviours regarding bike collection and return.
    It employs clustering and classification techniques to build predictive models, addressing key questions to boost operational efficiency: Which stations face the highest demand,
    and how do these patterns change throughout the day?
    Are there clear trends in user destination preferences, and when do stations experience peak activity?
    This investigation also identifies periods requiring urgent bike redistribution to meet high demand and defines `peak hours` based on empirical data analysis.

    \section*{Dataset description}
    \label{sec:dataset}

    This dataset provides anonymised historic trip data from the Divvy bike-sharing service, detailing user travel patterns across various stations and times.
    It captures the dynamics of bike usage, offering insights into the frequency and distribution of trips.
    Our analysis primarily focuses on the start and end stations' IDs to identify which stations require re-stocking and to infer the most popular destinations.
    Additionally, correlating trip start and end times with station usage is essential for a comprehensive understanding.\newline

    The dataset includes the following key features:\newline

    \subsection{Dataset Features}\label{subsec:dataset-features}
    \begin{itemize}
        \item \textbf{star\_ride\_id}: A unique identifier for each trip, ensuring individual trips can be tracked and analyzed discretely.
        \item \textbf{star\_rideable\_type}: The type of bike used for the trip, which can influence usage patterns and availability needs.
        \item \textbf{star\_started\_at}: The timestamp indicating when a trip started, crucial for understanding demand over time.
        \item \textbf{star\_ended\_at}: The timestamp indicating when a trip ended, allowing for the calculation of trip duration and temporal patterns of bike usage.
        \item \textbf{star\_start\_station\_name}: The name of the station where the trip originated, providing a geographic point for demand analysis.
        \item \textbf{star\_start\_station\_id}: A unique identifier for the origin station, which can be used in conjunction with geographic data for mapping and spatial analysis.
        \item \textbf{star\_end\_station\_name}: The name of the station where the trip concluded, indicating the destination demand in the network.
        \item \textbf{star\_end\_station\_id}: A unique identifier for the destination station, useful for spatial analysis and redistribution strategies.
        \item \textbf{star\_start\_lat \& start\_lng}: The latitude and longitude of the start station, giving precise location data for origin points.
        \item \textbf{star\_end\_lat \& end\_lng}: The latitude and longitude of the end station, giving precise location data for destination points.
        \item \textbf{star\_member\_casual}: A categorization of the user as a member or a casual rider, which can influence riding patterns and frequency of use.
    \end{itemize}

    \subsection{Added Features}\label{subsec:added-features}
    \begin{itemize}
        \item \textbf{week\_day\_index}: Categorised into Morning, Afternoon, Evening, and Night, to analyse demand fluctuations.
        \item \textbf{day\_period\_index}: Classified into working and non-working days, to observe weekly patterns in bike usage.
    \end{itemize}

    While the focus is not on text processing, storing text-based features like station names and user types enriches data visualisation, making it more accessible and interpretable.
    This approach, although not critical for data model training, enhances the user experience during data exploration.\newline
    \newline\textbf{Importance of Accessibility:}\newline
    A dataset's value is significantly enhanced by its accessibility.
    By detailing and explaining the dataset's features, this report aims to ensure that the data is as useful and informative as possible for the intended analysis.

    \section*{Method}
    \label{sec:method}

    \subsection{Data Acquisition}\label{subsec:data-acquisition}

    The data acquisition process marks the foundational step of our analysis, entailing a systematic approach to sourcing data directly from the designated endpoint.
    This task is accomplished through a combination of web scraping and automated data handling procedures, all orchestrated within a notebook environment to ensure efficiency and reproducibility.

    The procedure initiates with an HTTP request to the relevant endpoint, targeting the acquisition of zip files that contain the dataset.
    Upon successful download, these zip files are programmatically extracted to reveal the CSV files nested within.
    This process not only streamlines the retrieval of valuable data but also prepares it for subsequent stages of analysis.

    Following extraction, the CSV files are stored locally within a conventional file storage system.
    This choice of storage facilitates easy access and manipulation in later phases of the project.
    The final step in the data acquisition process involves loading the data into PySpark.
    This transition is meticulously governed by a predefined YAML schema, ensuring that the dataset's structure aligns perfectly with our analytical framework.
    By adhering to this schema, we guarantee the integrity and consistency of the data, laying a robust foundation for the comprehensive analysis that follows.

    This methodical approach to data acquisition not only secures the necessary dataset for our investigation but also exemplifies a scalable and repeatable model for similar data-driven projects.
    Through the integration of automated scraping, efficient data handling, and structured loading techniques, we establish a solid baseline from which to explore the intricacies of bike-sharing system optimization.
    the source code is available at~\cite{TechProjectSourceCode}.

    \subsection{Feature engineering}\label{subsec:feature-engineering}
    The feature engineering phase of our analysis is pivotal for simplifying the dataset, enabling more nuanced exploration and enhancing the predictive model's performance.
    By introducing categorical fields for the time of day and day of the week, we aim to dissect the data further, identifying usage patterns across different times and categorizing days into workdays and non-working days.
    This categorization simplifies the analysis and lays the groundwork for more in-depth insights into trip frequencies and preferences, crucial for optimizing bike-sharing services.\newline
    \textbf{Added Features:}\newline
    \textbf{Time of Day}:\newline We categorize each trip into one of four periods: Morning, Afternoon, Evening, and Night. This classification helps in analysing station popularity and bike usage trends at different times.\newline
    \textbf{Day of Week}:\newline Trips are classified into Workday or Non-Working Day categories, facilitating an understanding of how commuting patterns vary between weekdays and weekends.\newline

    \textbf{time of Day and Day of Week Categorization}\newline
    \begin{itemize}
        \item \textbf{Data Cleaning and Transformation}:\newline This step is essential for maintaining the integrity and accuracy of our findings. In preparation for sophisticated analyses, we rigorously clean the dataset by removing duplicates and null values.
        \item \textbf{Rideable Type Categorization}:\newline We categorize `rideable\_type` into three groups: classic bike (0), docked bike (1), and electric bike (2).
                        For the purpose of our predictive analysis—focusing on the demand for bikes at different times and locations—we opt to exclude the 'docked\_bike' category.
                        This decision is based on its limited relevance to the study's objectives, streamlining the dataset for more targeted analysis.

%        \item \textbf{Leveraging PySpark and User-Defined Functions}
%        \item \begin{itemize}
%                  \itemTo transform the \textbf{started_at} timestamps into our categorical features, we leverage PySpark functions alongside user-defined functions.
%                  This approach allows us to efficiently generate new categorical columns from existing data.
%        \end{itemize}
%        \item \textbf{StringIndexer Conversion}
%            \item \begin{itemize}
%                \item Following the creation of categorical features, we employ PySpark's StringIndexer to convert these categories into numeric values.
%                This conversion is crucial for incorporating these features into our predictive models, ensuring that the data is in a format that can be easily processed and analysed.
%          \end{itemize}
    \end{itemize}

    \subsection{Exploratory Data Analysis (EDA)}\label{subsec:analytical-techniques}
%    \paragraph{Overview}
%    The EDA phase is a critical component of the project, aimed at thoroughly examining the cleansed dataset to uncover underlying patterns, trends, and correlations.
%    This step is foundational for informing subsequent model training and refinement processes.
%
%    \subparagraph{Strategies and Tools}
%    \textbf{PySpark for Data Preparation}:\newline Initially, the dataset undergoes preparation and cleaning within the PySpark environment.
%    This step is crucial for handling large volumes of data, leveraging PySpark's distributed computing capabilities to efficiently process and ready the data for analysis.
%
%    \textbf{Transition to Pandas for Analysis}:\newline Once the data is prepped, it's converted into a Pandas dataframe.
%    This conversion is instrumental for the detailed analysis phase, where the rich functionalities of Pandas and its compatibility with visualization libraries come into play.
%
%    \subparagraph{Analytical Techniques}
%    \textbf{Aggregation and Summarization}:\newline
%    Key features of the dataset are aggregated and summarized to distill essential insights.
%    This involves counting occurrences, calculating averages, and other statistical measures to better understand the dataset's characteristics.
%
%    \textbf{Visualization}:\newline A variety of graphical representations—ranging from basic graphs and tables to more complex heatmaps—are employed to visualize the data.
%    These visual aids are invaluable for identifying patterns, detecting anomalies, and understanding the relationships between different dataset features.
%
%    Graphs and tables offer a straightforward means of presenting quantitative information, making it easier to grasp the distribution and variation within key variables.
%    Heatmaps are particularly useful for revealing correlations and patterns across multiple variables, providing a visual summary of complex relationships in a clear and concise manner.
%    \section*{Predictive Models}\label{sec:predictive-models}
%    \begin{itemize}
%        \item Two predictive model approaches are proposed and will be evaluated. This can change as the project evolves.
%        \begin{itemize}
%            \item  A \textbf{classification or clustering model} to identify patterns and groupings within the data.
%            \item A \textbf{time series forecasting or regression analysis model} to predict future trends based on historical data.
%        \end{itemize}
%        The project will explore both approaches. Comparing performance and best fit to the context of this Analysis.
%        \item     \end{itemize}
%    \item \textbf{Performance Review:}
%    \begin{itemize}
%        \item Review the performance of each predictive model.
%        Through using a set of metrics appropriate to each model type.
%        \newline This could include accuracy, precision, recall, and F1 score for classification models, or mean absolute error (MAE), mean squared error (MSE), and R-squared for regression models.
%    \end{itemize}

    \subsection{Clustering Analysis}\label{subsec:clustering-analysis}

    \subsection{Classification Analysis}\label{subsec:classification-analysis}

    \section*{Conclusion}

    \subsection{Summary}\label{subsec:summary}
    Pending Text

    \subsection{Future Work}\label{subsec:future-work}
    \begin{itemize}
        \item \textbf{Geodict Coordinates} Incorporating geographic data to identify areas with high demand and low supply.
        \item \textbf{Temporal Analysis:} Identifying peak hours and periods of high demand to inform redistribution strategies.
        \item \textbf{Predictive Modelling:} Developing models to forecast demand and supply, enabling proactive redistribution.
    \end{itemize}
    \begin{thebibliography}{24}

        \bibitem{TechProjectSourceCode}
        D. Pessoa, ``TechProject1: Source Code for Bike Sharing System Analysis,'' 2024.
        [Online].
        Available: \url{https://github.com/diogo-pessoa/TechProject1}.
        [Accessed: 07-02-2024].

        \bibitem{DivvyData}
        DivvyData, ``Bike ride-sharing data,'' 2023.
        [Online].
        Available: \url{https://divvybikes.com/system-data}

        \bibitem{DataSource}
        DivvyData Dataset, ``Index file for historic data,'' 2023.
        [Online].
        Available: \url{https://divvy-tripdata.s3.amazonaws.com/index.html}

        \bibitem{2016}
        J. Zhang, X. Pan, M. Li, and P. S. Yu, ``Bicycle-Sharing System Analysis and Trip Prediction,'' in \textit{2016 17th IEEE International Conference on Mobile Data Management (MDM)}, vol.
        1, pp.
        174-179, 2016.
        DOI: \href{https://doi.org/10.1109/MDM.2016.35}{10.1109/MDM.2016.35}

        \bibitem{2015}
        Y. Li, Y. Zheng, H. Zhang, and L. Chen, ``Traffic prediction in a bike-sharing system,'' \textit{Proceedings of the 23rd SIGSPATIAL International Conference on Advances in Geographic Information Systems}, SIGSPATIAL '15, New York, NY, USA: Association for Computing Machinery, 2015, Art.
        no.
        33.
        DOI: \href{https://doi.org/10.1145/2820783.2820837}{10.1145/2820783.2820837}

        \bibitem{s18020512}
        F. Chiariotti, C. Pielli, A. Zanella, and M. Zorzi, ``A Dynamic Approach to Rebalancing Bike-Sharing Systems,'' \textit{Sensors}, vol.
        18, no.
                2, Art.
                no.
                512, 2018.
                DOI: \href{https://www.mdpi.com/1424-8220/18/2/512}{10.3390/s18020512}

    \end{thebibliography}
\end{document}
