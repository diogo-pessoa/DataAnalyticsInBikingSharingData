\subsection{Conclusion and Summary}\label{subsec:summary}

This report embarked on an in-depth exploration of optimizing bike re-stocking within the dynamic context of urban mobility and technological evolution.
By delving into the complexities of bike-sharing systems, particularly focusing on the challenges associated with ensuring optimal bike availability, we have addressed critical aspects of this multifaceted problem.
Our investigation was informed by a thorough review of existing research, highlighting the significant strides made in this field, yet also identifying gaps where further innovation could yield substantial improvements.

\subsection{Addressing the Research Questions}\label{subsec:addressing-the-research-questions}
The core inquiries set forth at the beginning of this report centred around identifying patterns of bike usage across different times and stations, optimizing the distribution of bikes to meet demand, and enhancing operational efficiency through predictive modelling.\newline Through our analyses, we have:\newline
\begin{itemize}
    \item Identified the most frequented stations and discerned clear patterns in bike collection and return, particularly in relation to time of day and week.
    \item Utilized clustering and regression analyses to predict demand fluctuations, thereby informing strategies for bike distribution.
    \item Explored the integration of real-time traffic data and advanced temporal granularity to refine the precision of our predictive models.
\end{itemize}

\subsection{Key Learnings}\label{subsec:key-learnings}
Our journey through data exploration, feature engineering, and predictive modelling has yielded several key insights:\newline
\begin{itemize}
    \item  \textbf{The importance of temporal granularity}: Increasing the resolution of time segments to as short as 15-minute intervals significantly enhances the accuracy of demand predictions, facilitating more precise resource allocation.
    \item  \textbf{The value of real-time data integration}: Incorporating live traffic information into re-stocking route planning can drastically improve operational efficiency, ensuring that bikes are redistributed in a timely and effective manner.
    \item  \textbf{The potential of leveraging geodetic coordinates}: Analysing trip start and end points at a granular level opens new avenues for optimizing station stocking and uncovering user preferences for specific destinations.
\end{itemize}

In conclusion, this report addresses the initial research questions and contributes to a more in-depth understanding of the operational challenges and opportunities within bike-sharing systems.
The insights and methodologies developed herein offer a robust framework for enhancing service efficiency, ultimately contributing to the broader goals of sustainable urban mobility and technological innovation in public transportation systems.

\subsection{Future Work and Findings}\label{subsec:future-work}
Looking ahead, the findings of this report pave the way for countless promising areas of future research and operational enhancements:\newline
\begin{itemize}
    \item \textbf{Enhanced Route Optimization}: Employing cartographic tools and real-time traffic data to develop dynamic routing algorithms for bike re-stocking, potentially transforming the logistics of bike-sharing operations.
    \item  \textbf{User Engagement Strategies}: Investigating the impact of financial incentives for users who contribute valuable data, such as preferred drop-off points, could accelerate the optimization of the bike re-stocking process.
    \item  \textbf{Adaptation to Emerging Urban Mobility Trends}: As new ride types, including electric bikes, continue to gain popularity, adapting predictive models and operational strategies to accommodate these changes will be crucial.
\end{itemize}

\begin{thebibliography}{24}

    \bibitem{TechProjectSourceCode}
    D. Pessoa, ``TechProject1: Source Code for Bike Sharing System Analysis,'' 2024.
    [Online].
    Available: \url{https://github.com/diogo-pessoa/TechProject1}.
    [Accessed: 07-02-2024].

    \bibitem{DivvyData}
    DivvyData, ``Bike ride-sharing data,'' 2023.
    [Online].
    Available: \url{https://divvybikes.com/system-data}

    \bibitem{DataSource}
    DivvyData Dataset, ``Index file for historic data,'' 2023.
    [Online].
    Available: \url{https://divvy-tripdata.s3.amazonaws.com/index.html}

    \bibitem{2016}
    J. Zhang, X. Pan, M. Li, and P. S. Yu, ``Bicycle-Sharing System Analysis and Trip Prediction,'' in \textit{2016 17th IEEE International Conference on Mobile Data Management (MDM)}, vol.
    1, pp.
    174-179, 2016.
    DOI: \href{https://doi.org/10.1109/MDM.2016.35}{10.1109/MDM.2016.35}

    \bibitem{2015}
    Y. Li, Y. Zheng, H. Zhang, and L. Chen, ``Traffic prediction in a bike-sharing system,'' \textit{Proceedings of the 23rd SIGSPATIAL International Conference on Advances in Geographic Information Systems}, SIGSPATIAL '15, New York, NY, USA: Association for Computing Machinery, 2015, Art.
    no.
    33.
    DOI: \href{https://doi.org/10.1145/2820783.2820837}{10.1145/2820783.2820837}

    \bibitem{s18020512}
    F. Chiariotti, C. Pielli, A. Zanella, and M. Zorzi, ``A Dynamic Approach to Rebalancing Bike-Sharing Systems,'' \textit{Sensors}, vol.
    18, no.
            2, Art.
            no.
            512, 2018.
            DOI: \href{https://www.mdpi.com/1424-8220/18/2/512}{10.3390/s18020512}

\end{thebibliography}
